\documentclass{yokou}

\pagestyle{empty}

%\課題{卒業論文試問予稿}
\課題{修士論文試問予稿}
\daimei{ナノフォトニック・デバイスを用いた\\配列アラインメント用レースロジック回路の提案} 
% デフォルトは自動改行.\\ で強制改行.
\発表者{浅井 里奈}
\教官{井上 弘士教授}
\日付{平成 30 年 2 月 20 日}
\時間{13 : 30 〜 13 : 50}
\場所{大講義室(W2-313)}
\univ{九州大学}
\dept{大学院システム情報科学府}
%\dept{工学部 電気情報工学科}
\divn{情報知能工学専攻}
\divn{社会情報システムコース}

\begin{document}
%800文字以内に抑える.22行程度.
\begin{contentyokou}
生物配列(DNAの塩基配列やタンパク質アミノ酸配)
同士の類似度を求める配列アラインメントという手法がある.
生物配列の量は膨大で,その処理には多くの時間を要する.
そのため,ハードウェアアクセラレータを用いた配列アラインメントの高速化に関する
研究が多く行われてきた.
その一つに,
``レースロジック”がある.
レースロジックは,信号が回路に入力されてから出力されるまでの
伝搬遅延時間によって計算結果を表現するという特徴を持ち,
動的計画法によって解くことができる最適化問題の高速化が期待できる.
配列アラインメント処理への応用は,
CMOSトランジスタを用いた実装でその有効性が明らかにされている.
しかしながら,電気信号を伝搬に用いた場合,
配線のRC遅延によりその高速化が難しい状況である.

そこで本研究では,更なる高速化を実現する手段として
ナノフォトニクスという新しい光技術に着目し,
ナノフォトニック・デバイスを用いた
配列アラインメント用レースロジック回路を提案する.
また,提案回路に関して光学シミュレータを
用いた機能検証を行う.
遅延時間・面積・消費電力のモデルを構築し
評価を行った結果,
DNA配列長Nに対して,遅延時間はNに線形に従い,面積はNの2乗に従うことが明らかになった.
消費電力に関しては,ケーススタディとしてN=2の場合において
最低で0.144 mWにて動作することを示した.
また,雑音の信号伝搬に従って蓄積するという特徴や
光伝搬信号とナノフォトニック・デバイスの光速での計算能力,
光デバイス独自の設計選択肢について考察した.
その結果,
雑音が回路規模に及ぼす影響や,
光伝搬信号の光速での計算能力に追従する
回路伝搬遅延時間の検出感度を実現する方法を検討する
必要があることが明らかになった.
\end{contentyokou}
\end{document} 

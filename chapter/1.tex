\chapter{はじめに}
高性能と低消費電力を実現させるために,多くのアプリケーションにおいて専用アクセラレータが検討されてきた.
その中でも,動的計画法(Dynamic Programming, DP)によって解決されるような広範なクラスの最適化問題を高速化するために,
"Race Logic "と呼ばれる新しいコンピューティングのアプローチが提案された\cite{madhavan2014race}.
Race Logicの基本概念は,回路に設定された競合条件を使用して有用な計算を実行することである.

Race Logicはその実装において複数の設計選択肢がある.
しかしながら,現状では従来の相補型金属酸化物半導体(CMOS)テクノロジで実装可能な
Race Logicのみしかその有効性が明らかになっていない.

一方,ナノフォトニクスと呼ばれる新しい光素子技術が注目を集めている.
このナノフォトニクスを用いて機能を実現したデバイスをナノフォトニック・デバイスという.
ナノフォトニック・デバイスは光速度で演算を実現できる素子として注目されており,パタン検出機構や加算器などのアーキテクチ
ャの検討がされている.

本論文では,Race Logicの更なる高性能化を目的とし,ナノフォトニック・デバイスを用いた実装を提案する.
より具体的に評価を行うために,ナノフォトニック・デバイスを用いた光Race Logic(以下,光Race Logic)の性能をDNAグローバル配列アライメントタスクの例を用いて検討する.

本論文の構成は以下の通りである.
第2章でRace Logicの基本原理,及び実装に関する課題を説明する.
第3章では,光デバイスの基礎事項と共にナノフォトニクスの基本事項を説明し,光Race Logicの実装について述べる.
第4章ではケーススタディとして配列アラインメントを取り上げる.配列アラインメントの原理説明を行った後,設計選択肢の整理,回路の提案を行う.
第5章では提案した回路に対して評価と考察を行い,第6章でまとめを行う.


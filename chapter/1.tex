\chapter{はじめに}
現在,生物学の分野において生物配列(DNAの塩基配列とタンパク質アミノ酸配列) の文字列処理(配列情報解析)が注目されている.
DNAやタンパク質はユニットと名付けられた単位の物質が一列に並んだ高分子である.
ここでいうユニットとは,DNAにおいては4種の核酸,タンパク質においては20種類のアミノ酸である.
それぞれのユニットを文字としDNAやタンパク質の配列を単なる文字列だとみなして処理をしてもある種の本質は失われないという考えに基づき,文字列処理をすることで生物配列の解析を行なっている.
DNAの塩基配列やタンパク質アミノ酸配列の研究は,バイオインフォマティクスの最重要課題の一つとして取り組まれてきた.
配列情報解析の重要な対象であるゲノム塩基配列は,すでに200種類以上が決定されており,さらに多くの解析が進行中であるといわれている.
膨大なゲノム配列をこのように高速に決定できるようになった要因には,塩基配列の配列情報解析技術の進歩が挙げられる.
しかしながら,このゲノム解析にはさらなる高速化が求められており,目下課題となっている.

一方,ナノフォトニクスと呼ばれる新しい光素子技術が注目を集めている.
このナノフォトニクスを用いて機能を実現したデバイスをナノフォトニック・デバイスという.
ナノフォトニック・デバイスは光速度で演算を実現できる素子として注目されており,パタン検出機構や加算器などのアーキテクチャの検討がされている.

本研究ではDNA配列のグローバルアライメントを高速化する手段として,ナノフォトニクス・デバイスに着目した.
本論文の構成は以下の通りである.
第2章で配列アラインメントの基本原理,及びCMOSを用いたアラインメントアーキテクチャと問題点について説明する.
第3章では,光デバイスの基礎事項と共にナノフォトニクスの基本事項を説明し,設計選択肢を整理した上で回路の提案を行う.
第4章では提案した回路に対して評価と考察を行い,第5章でまとめを行う.

\cite{光コンピューテイング}

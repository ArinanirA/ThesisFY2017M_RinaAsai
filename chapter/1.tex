\chapter{はじめに}
高性能化や低消費電力を実現させるために,多くのアプリケーションにおいて専用アクセラレータが検討されてきた.
その中でも,動的計画法(Dynamic Programming, DP)によって解くことができる最適化問題を高速化するために,
"Race Logic"と呼ばれる新しいコンピューティングのアプローチが提案された\cite{madhavan2014race}.

Race Logicはその実装において複数の設計選択肢がある.
しかしながら,現状ではCMOSテクノロジで実装可能なもののみしかその有効性が明らかになっていない.

一方,ナノフォトニクスと呼ばれる新しい光素子技術が注目を集めている.
このナノフォトニクスを用いて機能を実現したデバイスをナノフォトニック・デバイスという.
ナノフォトニック・デバイスは光速度で演算を実現できる素子として注目されており,パタン検出機構や加算器などのアーキテクチャの検討がされている.

本論文では,Race Logicの更なる高性能化を目的とし,ナノフォトニック・デバイスを用いた実装を提案する.
より具体的に評価を行うために,ナノフォトニック・デバイスを用いた光Race Logic(以下,光Race Logic)の性能をDNAグローバル配列アライメントタスクの例を用いて検討する.

本論文の構成は以下の通りである.
第2章でRace Logic及び配列アラインメントの基本原理を説明し,
Race Logicの実装に関する課題を説明する.
第3章では,光デバイスの基礎事項と共にナノフォトニクスの基本事項を説明し,光Race Logicの実装を提案する.
第4章では提案した回路に対して検証と評価を行い,第5章でまとめを行う.


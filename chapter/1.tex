\chapter{はじめに}
高性能化や低消費電力を実現させるために多くのアプリケーションにおいて専用アクセラレータが検討されており,
その中で``レースロジック"と呼ばれる新しいコンピューティングのアプローチが提案された\cite{madhavan2014race}.
レースロジックは,回路を伝搬する信号の遅延時間がとある情報を表現しており,
入力から出力までの伝搬時間が計算結果を示すものである.
このアプローチは,動的計画法(Dynamic Programming)によって解くことができる
最適化問題の高速化が期待できる.
レースロジックの有効性は,CMOSトランジスタを用いて実装された
DNAのグローバル配列アラインメントスコアを求める回路にて明らかにされている\cite{madhavan2014race,madhavan2015race,madhavan20174}.

一方,ナノフォトニクスと呼ばれる新しい光素子技術が注目を集めている.
このナノフォトニクスを用いて機能を実現したデバイスをナノフォトニック・デバイスという.
ナノフォトニック・デバイスは光速度で演算を実現できる素子として注目されており,
ニューラルアクセラレータや加算器などのアーキテクチャの検討がされている\cite{kawakami,shisobe,Ishihara2016opti}.

本研究では光デバイスとレースロジックとの親和性に着目し,
ナノフォトニック・デバイスを用いたレースロジック回路の構成を目的とした.
本論文では,ナノフォトニック・デバイスを用いたDNAのグローバル配列アラインメントスコアを求める回路を提案し,
シミュレータを用いた回路の機能検証とモデル式を用いた評価を行うことでその有効性を明らかにし,実現に向けての問題点について検討する.

本論文の構成は以下の通りである.
第2章でレースロジック及び配列アラインメントの基本原理を説明し,
CMOSを用いて実装されたレースロジックに基づく回路例について述べる.
第3章では,光デバイスとナノフォトニクスの基本事項を説明し,
ナノフォトニック・デバイスを用いたレースロジックに基づく
DNAのグローバル配列アラインメントスコアを求める回路を提案する.
第4章では提案した回路に対してシミュレータを用いた検証とモデル式を用いた評価を行い,第5章で考察を行う.
最後に第6章でまとめを行う.


\chapter{はじめに}
高性能と低消費電力を実現させるために,多くのアプリケーションにおいて専用アクセラレータが検討されてきた.
その中でも,動的計画法(Dynamic Programming, DP)によって解決されるような広範なクラスの最適化問題を高速化するために,
"Race Logic "と呼ばれる新しいコンピューティングのアプローチが提案された\cite{madhavan2014race}.
Race Logicの基本概念は,回路に設定された競合条件を使用して有用な計算を実行することである.

Race Logicはその実装において複数の設計選択肢がある.
しかしながら,現状では従来の相補型金属酸化物半導体(CMOS)テクノロジで実装可能な
Race Logicのみしかその有効性が明らかになっていない.

一方,ナノフォトニクスと呼ばれる新しい光素子技術が注目を集めている.
このナノフォトニクスを用いて機能を実現したデバイスをナノフォトニック・デバイスという.
ナノフォトニック・デバイスは光速度で演算を実現できる素子として注目されており,パタン検出機構や加算器などのアーキテクチ
ャの検討がされている.

本論文では,Race Logicの更なる高性能化を目的とし,ナノフォトニック・デバイスを用いた実装を提案する.
より具体的に評価を行うために,ナノフォトニック・デバイスを用いた光Race Logic(以下,光Race Logic)の性能をDNAグローバル配列アライメントタスクの例を用いて検討する.

本論文の構成は以下の通りである.
第2章でRace Logicの基本原理,及び実装に関する課題を説明する.
第3章では,光デバイスの基礎事項と共にナノフォトニクスの基本事項を説明し,光Race Logicの実装について述べる.
第4章ではケーススタディとして配列アラインメントを取り上げる.配列アラインメントの原理説明を行った後,設計選択肢の整理,回路の提案を行う.
第5章では提案した回路に対して評価と考察を行い,第6章でまとめを行う.


\begin{comment}現在,生物学の分野において生物配列(DNAの塩基配列とタンパク質アミノ酸配列) の文字列処理(配列情報解析)が注目されている\cite{浅井潔2000配列情報と確立モデル,後藤修1998マルチプルアラインメントは生体高分子情報の交差点}.
DNAやタンパク質はユニットと名付けられた単位の物質が一列に並んだ高分子である.
ここでいうユニットとは,DNAにおいては4種の核酸,タンパク質においては20種類のアミノ酸である.
それぞれのユニットを文字としDNAやタンパク質の配列を単なる文字列だとみなして処理をしてもある種の本質は失われないという考えに基づき,文字列処理をすることで生物配列の解析を行なっている.
DNAの塩基配列やタンパク質アミノ酸配列の研究は,バイオインフォマティクスの最重要課題の一つとして取り組まれてきた.
配列情報解析の重要な対象であるゲノム塩基配列は,すでに200種類以上が決定されており,さらに多くの解析が進行中であるといわれている\cite{浅井潔2005バイオインフォマティクス}.
生物配列の文字列処理の中で,DNA配列中に同じ順序で並んでいるユニットのパターンを見つける配列アラインメントがある\cite{須戸里織2011バイオインフォマティクスゲノム配列から機能解析へバイオインフォマティクスゲノム配列から機能解析へ}.
アラインメントとは,複数の配列を入力として配列要素の間に最適な対応関係を求める処理であり,配列の類似性判定に応用できるものである.
配列アラインメントは生物学において重要な手法であり,計算機を用いた処理の高速化は従来よく多くの研究がなされてきた\cite{須戸里織2011gpu,宗川裕馬2008統合開発環境,sandes2011smith,liu2015accelerating,伊野文彦2007gpu}.

一方,ナノフォトニクスと呼ばれる新しい光素子技術が注目を集めている.
このナノフォトニクスを用いて機能を実現したデバイスをナノフォトニック・デバイスという.
ナノフォトニック・デバイスは光速度で演算を実現できる素子として注目されており,パタン検出機構や加算器などのアーキテクチャの検討がされている\cite{}.
\end{comment}

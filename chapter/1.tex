\chapter{はじめに}
高性能化や低消費電力を実現させるために多くのアプリケーションにおいて専用アクセラレータが検討されており,
その中で"Race Logic"と呼ばれる新しいコンピューティングのアプローチが提案された\cite{madhavan2014race}.
Race Logicとは,回路を伝搬する信号の遅延時間がとある情報を保持しており,
入力から出力までの伝搬時間を観測することで計算を行うことができる.
このアプローチは,動的計画法(Dynamic Programming)によって解くことができる最適化問題
の高速化させることができる.
Race Logicの有効性は,CMOSを用いて実装されたDNAのグローバル配列アラインメントスコアを求める回路にて明らかにされている.

一方,ナノフォトニクスと呼ばれる新しい光素子技術が注目を集めている.
このナノフォトニクスを用いて機能を実現したデバイスをナノフォトニック・デバイスという.
ナノフォトニック・デバイスは光速度で演算を実現できる素子として注目されており,パタン検出機構や加算器などのアーキテクチャの検討がされている.

本研究では光デバイスとRace Logicとの親和性に着目し,
ナノフォトニック・デバイスを用いたRace Logic回路の構成を目的とした.
本論文では,ナノフォトニック・デバイスを用いたDNAのグローバル配列アラインメントスコアを求める回路を提案し,
シミュレータを用いた回路の機能検証とモデル式を用いた評価を行うことでその有効性と実現に向けての問題点について検討する.

本論文の構成は以下の通りである.
第2章でRace Logic及び配列アラインメントの基本原理を説明し,
CMOSを用いて実装されたRace Logicに基づく回路例について述べる.
第3章では,光デバイスとナノフォトニクスの基本事項を説明し,
ナノフォトニック・デバイスを用いたRace Logicに基づく
DNAのグローバル配列アラインメントスコアを求める回路を提案する.
第4章では提案した回路に対してシミュレータを用いた検証とモデル式を用いた評価を行い,第5章で考察を行う.
最後に第6章でまとめを行う.


\chapter{考察}
前章にて,光Race Logic arrayの動作検証と
配列アラインメントスコア計算の高速化を見込めることを示した.
本章では,その性能や回路スケールに影響を及ぼす要因について考察していく.
\subsection{カウンターの分解能}
今回提案の回路では一つのセルの通過時間の差を検知できれば良い.
12.52ps 80GHz
遅延時間への重み付け ゲート長を1$\mu m$で調整できる
つまり10fsの遅延時間差も原理的には可能.
カウンタは100000GHz
そんな動作周波数のカウンタは現在存在しない
光Race Logic回路の性能を律速する要因たり得る.
array通過の遅延時間への情報付与に関しては実現可能なことは検証
光Race Logic arrayのポテンシャルを活かすカウンタ部分の構成を考えることが必要である.
\subsection{雑音の影響}
\subsubsection{光伝搬信号強度に影響を与える雑音}
検出可能な強度を担保する必要がある
光伝搬信号の特徴
雑音が素子を通るたびに蓄積していく
Nのスケーリングに影響を及ぼす
\subsubsection{遅延時間に影響を与える雑音}
製造ばらつきが要因になるものに代表される
素子の伝搬遅延誤差
伝搬遅延の誤差も素子を通るたびに蓄積
とある配列の組み合わせにおいて想定される遅延時間と
実際の出力の遅延時間に差が生じる
製造ばらつきであればある程度チューニングが可能か?
スコア1の重みをどの程度の遅延時間と定めるかによって
Nのスケーリングに影響を及ぼす.


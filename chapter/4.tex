\chapter{検証・評価}
本章では提案したDNA配列アラインメントスコア計算用の光Race Logic回路について機能検証と評価を行う.
\section{検証}
検証に用いたのはOptiwave社が提供するOptisystemというシミュレータである.
OptiSystemは光ネットワークのあらゆるタイプの広範囲のシステムの設計,評価,シミュレーションを行なうソフトウェアである.
素子レベルからシステムレベルまでの物理レイヤー上の光通信システムの設計と解析を行うことができる.

文字列長N=2の提案回路について,光Race Logic arrayの動作を確認した.
今回のシミュレーションにおいて,各素子において光伝搬信号に影響を与える雑音やロスは考慮していない.
またOptisystemの仕様上,遅延素子で付与された遅延時間のみが考慮され,
素子や導波路の伝搬遅延については考慮されていない.
その結果を図\ref{}に示す.
図\ref{}のnsnsの区間が文字列が完全一致の場合の出力,
nsnsの区間が文字列が一文字不一致の場合の出力,
nsnsの区間が文字列が完全不一致の場合の出力である.
それぞれにおいて,出力から最初に観測される信号のタイミングは入力から
ns後,ns後,ns後である.
図\ref{}を見ると,想定した動作をしているのが確認できる.

\section{評価}


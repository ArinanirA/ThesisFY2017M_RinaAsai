\chapter{おわりに}

ナノフォトニクスと呼ばれる新しい光素子技術と,
このナノフォトニクスを用いて機能を実現したナノフォトニック・デバイスが
光速度で演算を実現できる素子として注目されている.
本研究ではレースロジックと光デバイスとの親和性に着目し,
ナノフォトニック・デバイスを用いたアーキテクチャ検討の一環として
光レースロジックアレイを提案した.
また,市販されている光学シミュレータを用いて,雑音がない場合の
光レースロジックアレイの動作検証を行った.
更に,遅延時間・面積・消費電力についてのモデリングを行い,
配列長Nによって
それぞれの項目がどう変化するのかを示した.
遅延時間はNに線形に従い,面積はNの2乗に従うことが分かった.
消費電力に関しては,ケーススタディとしてN=2の場合の値を示した.

光レースロジック回路の性能を律速する要因として遅延時間差を検知する
部分の構造が重要になることを考察した.
検知部分は重み付けに用いる遅延時間の最小単位を検知できなければいけない.
ゲート長によって遅延時間が決定する光デバイスにおいて
ゲート長を1$\mu m$のサイズで加工するとき,
重み付けに用いる遅延時間の最小単位は10$fs$になる.
この遅延時間の最小単位を計測することができる検知部分の構成を考える必要がある.

また,光伝搬信号の強度や遅延時間に影響を与える雑音が
光レースロジックアレイの規模を律速しうると考察した.
雑音が光レースロジックアレイの規模を具体的にどう律速するのかを考えていかなければならない.

更に,出力信号の遅延時間に情報を付与するレースロジックの考えを発展させ,
光出力信号の位相や強度に情報を付与できる可能性も見えてきた.
光デバイス独自の設計選択肢が存在するのである.

ナノフォトニック・デバイスを用いた光レースロジック回路の実現に向けて,
雑音の影響を具体的に示すことと設計選択肢ごとの検知部分の構成が今後の課題である.

